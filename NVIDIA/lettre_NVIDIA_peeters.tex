\documentclass[12pt]{lettre}

\usepackage[utf8]{inputenc}
\usepackage[T1]{fontenc}
\usepackage{lmodern}
\usepackage{eurosym}
\usepackage[frenchb]{babel}
\usepackage{numprint}
\usepackage[a4paper,left=1cm, right=1cm, top=2.5cm, bottom=2.5cm]{geometry}

\begin{document}

\begin{letter}{Service des réclamations}
\name{Jean Râleur}
\address{Jean Râleur\\4, rue du Bac à sable\\80886 Sassone-le-Creux}
\lieu{Sassone-le-Creux}
\telephone{01 02 03 04 05}
\nofax

\def\concname{Objet :~} % On définit ici la commande 'objet'
\conc{Application for the NVIDIA Graduate Fellowship Program}
\opening{To whom it may concern,}

% qui est léo
% Diplome

After obtaining a double degree from \textit{IRCAM} and \textit{Télécom Paristech} in 2015, Léopold Crestel has started a Ph.D at IRCAM in September 2015 under my supervision. 
% Thèse
Léopold is working on artificial intelligence applied to music, which is currently one of the most prominent topic investigated in \textit{IRCAM} now. More specifically, its work focuses on developing a system able to automatically perform the orchestration of a piano score.
% Orchestration := c'est  chaud, ca fait flipper
Due to its tremendous complexity, orchestration as a scientific topic of interest has been avoided by the music computer community for a long time. 
% Ce qu'on fait = nouveau
The vast majority of the precedent work focused on discovering perceptively relevant audio descriptors.
% Ambitieux
No successful attempt to build a generative system producing symbolic data (i.e. scores) has been achieved.
% Mais ça va marcher
Hence, this is project is remarkably ambitious. But we believe that the artificial intelligence field has reached a sufficient point of development to tackle it.
% Work = Corner stone in the field
Through the results already obtained, we can foresee that his work will undoubtedly become a corner stone for the scientific investigation of orchestration : he created the most important database dedicated to this task, already proposed several \textit{GPU}-based algorithms obtaining state of the art results, and proposed thrilling axis of development for the future work.

% comment je le connais
I know Léopold since it's second year of Master of science as \textbf{his teacher in computer sciences}. He always demonstrated a profound interest for the different topics and a great capacity for going from the most abstract concept to their practical realisation.
% son investissement

% son rayonnement (lol) Dans le lab, c'est le boss du GPU.

% Pourquoi ce qu'il fait à besoin de race de GPU

\closing{Sincerely,}

\end{letter}

\end{document}
