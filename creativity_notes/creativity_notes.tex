\documentclass{report}
\usepackage[utf8]{inputenc}
\usepackage{framed}
\usepackage[outerbars]{changebar}

% Dimensions de la page
\usepackage{geometry}
\geometry{scale=0.8}
\geometry{a4paper}

%% Acronyms
%\usepackage[nonumberlist]{glossaries}
%\setacronymstyle{long-short}
%\makenoidxglossaries
%% Load acronyms list
%\loadglsentries{../acronyms}

% Graphics
\usepackage{graphicx}

% Prettyref
\usepackage{prettyref}
\usepackage{hyperref}

\title{Considerations on creativity}
\author{Léo Crestel}
\begin{document}
\maketitle

\section{Notes on \textit{Les coulisses de la création - Karol Beffa et Cédric Villani} \cite{Karol-Beffa:2015aa}}
Both outline the importance of the \textit{pastiche} game to develop its technical skills and understand how to develop a personal style through the analysis and imitation of other's.
But they both agree that imitating is necessary but not sufficient, as it is a mean to deeply understand what has been done and go further. Learning algorithms based on a predictive measure is like learning \textit{pastiching}, but what about this extension, the 'going further' part ?
\\
\textbf{ Learning = pastiche, creativity = ?}

Chapter 9 : \textit{Paul Valéry : Il faut être deux pour inventer. L'un forme les combinaisons, l'autre choisit, reconnaît ce qu'il désire et ce qui lui importe dans l'ensemble des produits du premier. Ce que l'on appelle génie est bien moins l'acte de celui-là, celui qui combine, que la promptitude du second à comprendre la valeur de ce qui vient de se produire et saisr ce produit}. \\
\textbf{Adversarial net !}. Or something like an generative system that proposes combinations and an other one that select them. -> Go further than adversarial nets.


%\nocite{*}
\bibliographystyle{alpha}
\bibliography{../biblio}

\end{document}